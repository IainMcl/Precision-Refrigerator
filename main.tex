 \documentclass[10pt]{article}
\usepackage[utf8]{inputenc}
\usepackage{natbib}
\usepackage{graphicx}
\usepackage[margin=0.75in]{geometry}
\usepackage{url}
\usepackage{amsmath}
\usepackage{comment}
\usepackage[utf8]{inputenc}
\usepackage[T1]{fontenc}
%\usepackage[font={small,it}]{caption}
%\usepackage[font={it}]{caption}
%\usepackage{fancyhdr}
%\pagestyle{fancy}
%\fancyhf{}
%\lfoot{Iain Mclaughlan}
%\rfoot{s1524154}

\usepackage{listings}
\usepackage{color}
 
\definecolor{codegreen}{rgb}{0,0.6,0}
\definecolor{codegray}{rgb}{0.5,0.5,0.5}
\definecolor{codepurple}{rgb}{0.58,0,0.82}
\definecolor{backcolour}{rgb}{0.95,0.95,0.92}

\title{Precision Refrigerator}
\author{Iain McLaughlan\\ s1524154 }
\date{\today}

\begin{document}

\maketitle
\begin{abstract}
A precision refrigerator was built and controlled using a raspberry pi to precisely control the temperature of wine (water) by toggling the on/off state of a Peltier\cite{peltier} cooling unit. Different methods for determining the on/off state of the Peltier heat pump were used to converged the temperature of the water on an aim temperature with different effectiveness at maintaining the temperature about the aim. A scoring system was devised to compare different convergence methods and the advantages and disadvantages of each were compared. It was found that for an aim temperature of $21.4^oC$: `Converge' held a mean temperature of $21.4(1)^oC$ with $53.2\%$ of the data falling withing the precision of the thermometer of the aim value. The more complex methods `Hysterisis Converge', `Rate Limit Convergence' and `Pre-emptive Convergence' had a means of $21.4(1)^oC$, $21.4(1)^C$ and $21.4(1)^oC$ respectively and percentage scores of $62.6\%$, $68.0\%$ and $73.0\%$. The efficiency of this cooling unit was then tested by comparing the energy consumed to change the temperature of the water with the specific heat capacity of water. The fridge was found to have a percentage efficiency of $6(1)\%$ when measured over $0.5$ and $1.0^oC$ temperature drops.
\end{abstract}

\section*{Introduction}
Microcontrollers can be used for a wide range of different applications. These applications in embedded systems and can range from driving robots\cite{robomicro} to burglar detection \cite{microburg}. A microcontroller refers to a small computer that is usually based on a single integrated circuit. This single circuit often contains a CPU, some form of memory and usually some form of input/output (I/O) peripherals. In recent times microcontrollers and in turn microprocessors have become significantly more powerful and can control much more complex systems as a result of this.\\

One application of a microcontroller is in the functioning of a precision refrigerator. Here a microcontroller can use various inputs to control the temperature of various objects. Knowing and being able to control the temperature of objects with a high level of precision is very important in different fields. For example, during experiments looking at reactions at phase boundaries\cite{microfluidic} being able to precisely control the temperature to keep the substance at the desired level is very important in obtaining reliable results.\\

To precisely control the temperature of a substance it is useful to understand how much energy is required to heat and cool it. The energy required to change the temperature of a substance is determined by the heat capacity. The heat capacity ($C$) of a substance is defined as the ratio of heat transferred ($Q$) to or from the system and the resulting change in temperature ($T$) of the substance.  % The heat capacity is measured in $kJ/kg K$.

\begin{equation}\label{eq:heat_cap}
    C(T) = \frac{\delta Q}{dt}
\end{equation}

The heat capacity changes depending on the substance that is being heated. On top of this, the heat capacity also varies with temperature and what variables are free to change and which are constrained i.e. constant pressure temperature changes vs constant volume temperature changes.\\

\begin{figure}[h!]
    \centering
    \includegraphics[scale=.75]{Heat_capacity_C.jpg}
    \caption{\it{The heat capacity of water varying differently when kept at constant pressure compared to being kept at constant volume\cite{heat_cap}.}}
    \label{fig:heat_cap_water}
\end{figure}

This variation can be seen in Figure \ref{fig:heat_cap_water} where the heat capacity of water varies differently depending on which state variable is kept constant (either pressure or volume). \footnote{At temperatures close to $0^oC$ the difference in heat capacities, Cp and Cv are approximately equal and only noticeably start to diverge from one another at temperatures close to $50^oC$.}\\

Using the energy required to change the temperature of the substance along with the amount of energy used in total can be used to calculate the efficiency of the cooling system. The efficiency of something is defined as the ratio of the useful work produced to the total energy expended. In the case of a refrigerator, this is the ratio of the energy supplied to the cooling unit compared to the energy actually used to cool.\\

\begin{figure}[h!]
    \centering
    \includegraphics[scale=.75]{ref.jpg}
    \caption{\it{Energy movement through a refrigerator system, transferring heat (Qc) from a cold source ($T_{cold}$) by providing work evergy (W) to drive the heat (Qh) to the hot sink ($T_{hot})$\cite{fridge}.}}
    \label{fig:fridge}
\end{figure}

Refrigerators operate by supplying work to move heat from a cold source to a hot sink. The flow of heat for this process can be seen in Figure \ref{fig:fridge}. The efficiency ($\eta$) of the refrigeration process is given by;
\begin{equation}\label{eq:eff}
    \eta = \frac{Q_C}{Q_H-Q_C}=\frac{Q_C}{Q_H/Q_C - 1}
\end{equation}
The ideal circumstance, where all of the work available is used to move heat, is given by the Carnot efficiency\cite{carnot}. In this idealised case, the efficiency is purely determined by the temperature ($T$) of the heat reservoirs.
\begin{equation}\label{eq:carn_eff}
        \eta_c = \frac{T_C}{T_H-T_C}=\frac{T_C}{T_H/T_C - 1}
\end{equation}
This is an idealised case describing the maximum efficiency, in reality it is impossible due to heat loss and friction to reach this efficiency. 


% RPI Not a micro processor or micro controller but system on chip


\section*{Aims}
\begin{itemize}
    \item Build a circuit that measures the temperature of a substance and can control the on/off state of a Peltier heat pump\cite{peltier} to control the cooling of the substance. 
    \item Measure the cooling efficiency of the cooling circuit when cooling a known volume of water by a known amount.
    \item Construct several convergence methods which determine the on/off state of the Peltier depending on given inputs and compare their abilities at precisely controlling the temperature of the water.
\end{itemize}
\section*{Method}
The Peltier heat pump cooling chip was controlled using a breadboard circuit built using a Raspberry Pi\cite{rpi} as a system on a chip microcontroller. The chip was powered using an external power supply which was connected and disconnected using a transistor. A python script was written to control the state of the transistor depending on temperature readings from a thermometer. \\

\subsection*{Thermometer Calibration}

The circuit was initially set up with a Programmable Resolution 1-Wire Digital Thermometer (DS18B20)\cite{thermometer}. This thermometer has a programmable precision of 9 to 12 bits and an accuracy of $\pm 0.5^oC$. In 12 bit mode, this gives the thermometer a highest precision where it is able to read at $2^{-4o} C (0.0625^o C)$. This was tested with a short python script which reads the temperature of the thermometer every 2 seconds\ref{app:therm_test}. A second thermometer was connected in parallel with the first and was tested in the same way. Over a period of 60 seconds with the thermometer probes measuring the room air temperature near to one another, the temperature from both thermometers was recorded. The temperatures were found to be consistent with one another with only small fluctuations which averaged out at the same value for both instruments. Since both probes measured the same temperature there was no systematic offset needed to be built in to further measurements. If a systematic error was found then an average of the two probes would have been found and the difference between the mean of one probe and the mean of both would be added to all further readings.\\

\subsection*{Peltier Setup}
The Peltier was connected to a GwInstek GPS-x3303 power supply\cite{powe_sup}. This was in turn connected to a Darlington T0220 Transistor\cite{trans} which acted as a switch and could be controlled by a Raspberry Pi by toggling the state of one of its GPIO (general purpose input output) pins between a high and a low voltage. When the input voltage to the transistor was high the transistor switched on which completed the Peltier circuit activating the cooling chip using the power supply. When the voltage was low the transistor switched off which disconnected the Peltier circuit and stopped the cooling chip from receiving power.\\

\begin{figure}[h!]
    \centering
    \includegraphics[scale=0.5]{circuit.png}
    \caption{\it{Circuit diagram showing the Peltier chip being powered by a DC power supply and controlled by a transistor switch whos state is determined by a RPi\cite{course_notes}.}}
    \label{fig:circuit}
\end{figure}

The cooling chip circuit was tested by toggling on and off the transistor with the power supply set to a low voltage and current. The chip was held to check that one side heated as the other side cooled when powered. This was only done in short bursts as leaving the chip running for long periods of time when not connected to a heat sink can damage the chip. \\

Once the Peltier and thermometer were working and tested the Peltier chip was placed on a heat sink. On top of this was placed a small beaker filled with $50ml$ of water and the thermometer probe was placed in the water to measure the temperature of the water. A second thermometer was set up to read the ambient room temperature near the water beaker. To help gain an understanding of how the water heats and cools a prolonged cooling phase ($\sim15$ minutes) was recorded and plotted to see the trends. Following this, the water was then left to heat for the same time and was again plotter to see the trends.


 % **** Set up diagram? ****

\subsection*{Controlling the Cooling Chip}
Several methods were used to determined the state of the cooling chip, which aimed to converge the temperature of the water down to a target temperature. In all methods, a precision range was set which corresponded to the precision with which the thermometer could measure (in some cases this was increased to see different responses). This was done to allow for inaccuracies in the temperature readings, prevent the cooling chip from rapidly switching on and off which could damage it, and provide an acceptable range around the air temperature which would be counted as the same as the air temperature (aim temperature $\pm$ precision = aim temperature).\\

These methods worked using the following rules:\\ % converge(), hysteretic_conv(), rate_limit_conv(), pre_empt_conv()

1 (Converge). If the temperature of the water was above the aim temperature plus the precision then the chip was switched on. Alternatively, if the temperature of the water was below the aim temperature minus the precision then the chip was turned off.\\

2 (Hysteretic$\_$conv). This method works by realising that since the ambient temperature is taken to be greater than the cooled temperature then the heating of the water will happen faster than the cooling process. To combat this the cooling chip was activated when the temperature was below the aim temperature minus half of the precision. This reduces the relative heating phase.\\

3 (Rate$\_$limit$\_$conv). It is assumed that the temperature of the room is suficently higher than the aim temperature such that the rate of heating from the ambient temperature will be greater than the rate of cooling due to the cooling chip. On this assumption the high end temperature limit where the chip is turned on is adaptivly moved depending on the difference between the ambient temperature and the aim temperature. If the room temperature is much higher than the aim temperature then the assumption is that the rate of heating will be very fast so the chip is turned on at a lower temperature than it would if the temperature difference was less. This should limit the over shoot of the temperature in the heating phase by changing the upper limit depending on the expected rate of temperature increase from the suroundings. \\

4 (Pre$\_$empt$\_$conv). Here the rate of temperature change is used to try and pre-empt the point at which the temperature will become greater than or less than the aim temperature and toggle the state of the cooling chip accordingly. This is done using the idea that the temperature of the water will continue to change even after the state of the chip has been changed. For example, when the water is heating, if the rate ($degrees/s$) added to the current temperature will be greater than the aim temperature then the cooling chip will be turned on to start the cooling process before the temperature has fully risen to the aim temperature. This rate can be multiplied by an estimated change factor i.e number of seconds for an effect to be realised, to try and work closer to the aim temperature.\\

To compare the different convergence methods a scoring system was devised which looks at temperature measurements over a test period and finds the fraction of these which were within the precision range of the desired temperature. The test range was chosen to only start when the temperature had first reached the aim temperature so as to only count the region around an aim temperature and not while initially cooling from the original temperature to the aim.   \\

In addition to this, the efficiency of the cooling system was measured. This measurement was made during the initial cooling phase before the aim temperature had been reached for the first time. This was done by recording the time ($t$) between when the chip was first turned on and when the aim temperature was reached. This time was multiplied by the power provided from the power source ($P=IV$) to give a total energy used to cool. This was then compared to the energy required to cool the water by the same temperature difference ($\Delta T$) using the known specific heat capacity ($c$) and mass ($m$) of the water. The ratios of these energies was used to find the efficiency of the overall cooling system:
\begin{equation}
    \eta_{sys} = \frac{EnergyToCoolWater}{TotalEnergySupplied} = \frac{cm\Delta T}{Pt}
\end{equation}


% using the heat capacity of water at room temp
\section*{Results and Discussion}
An important thing to note throughout this results section is that the precision of the thermometer in its most precise setting is $0.0625^oC$. This precision is the reason behind the step like nature in the raw data. Attempts were made to smooth the data which can be seen in Appendix \ref{app:smoothed} however this was felt to potentially over or under exaggerate changes in aiming to display smoother data. Instead of smoothing data the decision was made to run experiments over long periods of time which will show trends over the step-like nature of the temperature readings. \\

\subsection*{Thermometer Calibration}
%Thermometer calibration graph.
\subsection*{Cooling and Heating Data}
The cooling and heating data was recorded over 2000 data points which can be seen in Figure: \ref{fig:heat_and_cool_curve}. These show an exponential decay when cooling and an exponential increase when heating. Throughout the recording process, the ambient lab temperature steadily increased from $22.62^oC$ to $24.27^oC$ due to the presence of others working in the same room. Note, the water temperature is consistently slightly below room temperature even after being left for a long period of time.\\

\begin{figure}[h!]
    \centering
    \subfloat{{\includegraphics[width=8.45cm]{Cooling_data_curve_long.png} }}%
    \qquad
    \subfloat{{\includegraphics[width=8.45cm]{Heating_data_curve_long.png} }}%
    \caption{\it{Cooling and heating curves consisting of 2000 data points each. Each data point corresponds to $\sim 1$ second ($0.89$ seconds). }}%
    \label{fig:heat_and_cool_curve}%
\end{figure}

Cooling and Heating data recorded over 1000 data points rather than 2000 can be seen in Appendix \ref{app:short_data}. Comparing these shows that it takes a significant time for the temperature to reach its limiting point where the graph begins to plateau. Over short periods of time ($\sim 1$ minute) the rate of temperature change is constantly varying.\\

From the cooling data, it can be seen that the Peltier heat pump chip can only cool the water by around $1.8^oC$ from its initial temperature. This provides slight limitations later as operating at lower temperatures could give the possibility of working with a heating curve that is approximately linear over the working range. This could allow predictions to be made more readily of how quickly the water will heat. The same can not be done in reverse i.e. having a higher operating temperature to have an approximately linear cooling line due to the limitations of relying on the room temperature to act as a heater. When working close to the room temperature the relative heating is very slow and not reliable. Having either a larger or more powerful cooling chip could allow work at lower temperatures and make these approximations possible. \\

A further possible use for this data would be to use the curves to predict how quickly the temperature will change when at a given temperature. This is not possible in this instance due to the room temperature not being constant throughout. Even if the data was corrected to represent degrees below room temperature this would still not work as the efficiency of the cooling chip will change depending on the working temperatures. The Peltier cooling chip's efficiency will be limited due to the principles from equations \ref{eq:eff} and \ref{eq:carn_eff}. To control this the temperature difference between the bottom of the water beaker and the temperature of the top of the heat sink would have to be relatively constant. On top of this, the Peltier chip will have its own working efficiency which will limit this method of temperature change predictions.

\subsection*{Convergence Methods}
Through out each of these method tests measurements were made over 500 time steps (which correspond to about 1 second each). For each method the average temperature was found. A reduced $\chi^2$ value was obtained by comparing to an optimum straight line fit at the optimum temperature. And a percentage score for what percentage of the data points fall within one reading error of the thermometer from the aim temperature (aim temperature $\pm 2^{-4}$). For each set of readings the aim temperature was set to $21.4^oC$ and the starting room temperature for each was recorded.\\

\subsection*{Converge}
Here the cooling chip was toggled on and of if the water temperature was above or below the aim temperature of $21.4^oC$. \\

\begin{figure}[h!]
    \centering
    \includegraphics[scale=0.75]{converge.jpg}
    \caption{\it{The steady state oscillations of the `Converge' method about an aim temperature of $21.4^oC$.}}
    \label{fig:conv}
\end{figure}

It can be seen that the temperature of the water oscillates about the aim temperature in a sinusoidal manner where the temperature overshoots the aim temperature fairly evenly when both increasing and decreasing in temperature. \\

The average temperature over the 500 measurements was $21.4(1)^oC$. A total error was found by combining the thermometer precision with the standard deviation about the mean ($\sigma$), giving a value of the total error to be, $0.14^oC$. The reduced $\chi^2$ of the data when fitted to an ideal temperature, a horizontal line at the aim temperature was found to be, $0.29$ which is significantly less than 1 leading to the assumption that the data has been over fitted relative to the error. This method of convergence maintained the water temperature within the precision level of the thermometer of the aim temperature $53.2\%$ of the time \footnote{Examples of the calculations used to find these values can be seen in Appendix \ref{app:errors}.}.\\

\subsection*{Hysteresis Convergence}
Here the point at which the cooling chip was turned on was brought down so that the heating time was reduced compared to the cooling time to try and combat the effect of the rate of heating being faster on average than the rate of cooling.\\ 

\begin{figure}[h!]
    \centering
    \includegraphics[scale=0.75]{hyst_conv.jpg}
    \caption{\it{The steady state oscillations of the `Hysteretic$\_$conv' method about an aim temperature of $21.4^oC$.}}
    \label{fig:hyst}
\end{figure}

This gave a mean value of $21.4(2)^oC$ which is equivalent to `Converge'. This method gave a reduced $\chi^2$ when fitted to the aim temperature of $0.55$ which indicates a larger spread of data when again compared to `converge'. The changes in the cooling and heating phases results in a smaller frequency of oscilation when compared to other methods. This could have benifits if the cooler used became less efficient when switched on and off rappidly but instead operates better when either on or off for more prolonged periods of time. This method gave a percentage score of $62.6\%$ which controls the temperature more precisly than the base line convergence method used above.\\

\subsection*{Rate Limit Convergence}
Similarly to the hysteretic method above the upper limit at which the cooling chip is turned on is actively changed dependent on the difference between the aim temperature and the current room temperature. If there was a big temperature difference with the room temperature being much greater then it was assumed that the heating rate would be relatively quicker than if this difference was low. To combat this the upper limit was set to be smaller with higher temperature differences and larger when the heating rate due to the ambient temperature was slow.\\

\begin{figure}[h!]
    \centering
    \includegraphics[scale=0.75]{rate_lim.jpg}
    \caption{\it{The steady state oscillations of the `Rate$\_$limit$\_$conv' method about an aim temperature of $21.4^oC$.}}
    \label{fig:rate}
\end{figure}

Here the average value was $21.4(1)^oC$ with a reduced $chi^2$ about the aim of $0.29$. These results are very comparable to the previous two methods however more significant differences can be found when looking at the percentage score which showed that the temperature was within the precision of the thermometer $68.0\%$ of the time. It can also be seen that the frequency of oscilation here is larger. Here there are approximatly 4.75 full oscilations in the 500 measurements compared to the approximatly 2.75 for `Converge' the reasoning behind this is unclear. A possible explanation is that the chip is being controled by a more advanced control system so the temperature is actualy being controled with more precision however this is not being fully shown due to the limitations of the precision of the thermometer.\\

\subsection*{Pre-emptive Convergence}
The state of the chip was controlled here by looking at the rate of temperature change over the last few steps and if that rate would carry the temperature over the aim temperature then the state of the Peltier was pre-emptivly toggled to the opposite state.\\

\begin{figure}[h!]
    \centering
    \includegraphics[scale=0.75]{pre_emp.jpg}
    \caption{\it{The steady state oscillations of the `Pre$\_$empt$\_$conv' method about an aim temperature of $21.4^oC$.}}
    \label{fig:pre}
\end{figure}

This method gave a mean value of $24(1)^oC$ and a reduced $\chi^2$ about the aim temperature of 0.24. The $\chi^2$ result is very simmilar to those seen previously along with a consistent average temperature across the 500 measurements. This method again steps on the percentage of data within the range about the aim to $73\%$. This is the only method where the improved convergence about the aim can be reasnoably seen by eye when comparing the graphs showing in Figures \ref{fig:conv}, \ref{fig:hyst}, \ref{fig:rate} and \ref{fig:pre}. Here the temperature on the high end sits on precision level lower than the other methods showing that the method keeps the temperature closer to the aim. A graph showing all of the plots overlayed on top of one another can be seen in Appendix \ref{app:comparison} where some of the differences between plots can be seen more easily.\\

Overall the convergence methods can be seen to be fairly comparabel when viewed by eye and only by looking deeper in to the effectivness of each method mainly through the use of the percentage score can a meaningful difference be seen. In the case where a cooling chip is used where the efficency of the chip may be compramised by toggling on and off more rapidly it may be benificial to use a convergence method similar to that of the `Hysteresis Convergence' due to the lowere frequency of oscilation of the temperatue. However if this is not an issue the other methods like the rate limiting and pre-emptive methods control the temperature about an aim with a higher level of precision. The pre-emptive method used was rough in the sence that assumptions were made in how long it would take for a change in state would take to make a difference to the temperature of the water. These values could be tuned through more trial runs to further improve this method.

\subsection*{Cooler System Efficiency}
The efficiency of the cooling system was measured by finding the ratio of the energy required to cool the water by the given temperature change to the energy supplied to the system from the power supply. These results were taken when cooling over different ranges from room temperature. throughout these calculations the heat capacity of water was taken to be constant at a value of $4.186KJkg^{-1}K^{-1}$\cite{heat_cap_val} and the input voltage and current from the power supply was $3.0V$ and $1.5A$ respectively. \\

\begin{table}[h!]
    \centering
    \begin{tabular}{|c|c|}
        \hline
        Efficiency ($\%$) & Cooling Range from room temperature ($^oC$) \\
        \hline
        8.24 & 0.5 \\
        4.00 & 0.5\\
        11.81 & 0.5 \\
        3.08 & 1.0\\
        3.65 & 1.0\\
        8.81 & 1.0\\
        \hline 
    \end{tabular}
    \caption{\it{Cooling system percentage efficiencies when cooled to a certain change below room temperature.}}
    \label{tab:efficiencies}
\end{table}

Here in table \ref{tab:efficiencies} it can be seen that the mean efficency of the system over all measurements was $6.60\%$ with a random error of $1.45\%$. The temperature drops when compared to room temperature were recorded as it was thought that because the cooling rate slows at lower temperatures which can be seen in Figure \ref{fig:heat_and_cool_curve} the efficency of the cooling unit would decrease when cooled to lower temperatures. This is shown to be correct by the efficiency of the cooling unit being significantly less when cooling over $1^oC$ when compared to cooling by $0.5^oC$. When the water was cooled by $1^oC$ the mean efficency was found to be $5(2)\%$. When cooled by $0.5^oC$ the mean efficency was $8(3)\%$.\\

The efficency of the overall system is very low. This is a result of many factors: The water that is cooled will gain heat from the surroundings at the same time. Not all of the cooling power of the Peltier will be transferred directly to the water. Some will go to the surroundings and some to the beaker. On top of this the heat of the temperature of the heat sink will increase with time as the cooling chip is on. This will reduce temperature difference between the hot and cold sinks. In turn this will reduce the maximum possible (Carnot, equation \ref{eq:carn_eff}) efficiency which will reduce the maximum realistic possible efficency observer of the cooling system. If the temperature of the heat sink was kept constant by the addition of a powered fan or another method then the maximum possible efficiency would not decrease with time. 


\subsection*{Discussion}

Therm calibration could be improved by measuring consistency over a range of temperatures.
Heat dispersion throughout the water.
Heating issues with heatsinks n that
score always starts on a down. therefore likely to be higher than actual.

\section*{Areas of Further Research}


\section*{Conclusion}

\bibliographystyle{unsrt}
\bibliography{references}



\newpage
\appendix
\section{Cooling and Heating Reference Curves} \label{app:short_data}
\begin{figure}[h!]
    \centering
    \subfloat{{\includegraphics[width=8.45cm]{Cooling_data_curve_short.png} }}%
    \qquad
    \subfloat{{\includegraphics[width=8.45cm]{Heating_data_curve_short.png} }}%
    \caption{\it{Cooling and heating curves consisting of 1000 data points each. Each data point corresponds to $\sim 1$ second ($0.89$ seconds). }}%
    \label{fig:heat_and_cool_curve}%
\end{figure}

\section{Error Analysis}\label{app:errors}
\subsection{Total Error}
\begin{equation}
    \sigma^2 = \sum \frac{(x_i - \mu)^2}{N}
\end{equation}
\begin{equation}
    \sigma_{total} = \sqrt{\sigma^2 + \sigma_{therm}^2}
\end{equation}
Here $\sigma_{total}$ is the combined error; $\sigma^2$, the standard deviation; and $\sigma_{therm}$ the precision of the thermometer which had a value of $0.0625^oc$.
\subsection{Reduced $\chi^2$}
\begin{equation}
    \chi_{red}^2 = \frac{1}{N-1}\sum \frac{(x_i - x_{expected})^2}{\sigma_{x}}
\end{equation}
Here N-1 represents the number of degrees of freedom and $x_{expected}$ the aim temperature for the given results set.
\subsection{Score}
\begin{equation}
    Score = \frac{\sum if |(x_i - x_{expected})| \leq \sigma_{therm}}{N}
\end{equation}

\section{Convergence Method Comparison}\label{app:comparison}
\begin{figure}[h!]
    \centering
    \includegraphics[scale=0.75]{comparison.jpg}
    \caption{\it{Here the convergence data for all of the convergence methods used can be seen together. This mainly shows the differences in the amplitudes of the different methods.}}
    \label{fig:comparison}
\end{figure}


\newpage
\section{Python Data Analysis Code}\label{ap:code}

\lstdefinestyle{mystyle}{
    backgroundcolor=\color{backcolour},   
    commentstyle=\color{codegreen},
    keywordstyle=\color{magenta},
    numberstyle=\tiny\color{codegray},
    stringstyle=\color{codepurple},
    basicstyle=\footnotesize,
    breakatwhitespace=false,         
    breaklines=true,                 
    captionpos=b,                    
    keepspaces=true,                 
    numbers=left,                    
    numbersep=5pt,                  
    showspaces=false,                
    showstringspaces=false,
    showtabs=false,                  
    tabsize=2
}
 
\lstset{style=mystyle}
 
% \begin{document}
% \lstinputlisting[language=python]{}
% \end{document}

\end{document}